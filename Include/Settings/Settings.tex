\usepackage[T1]{fontenc}		% Codifica dei font; T1 codifica di output dell'italiano e di lingue occidentali
\usepackage[utf8]{inputenc}		% Codifica degli input; abilita l'utilizzo di caratteri accentati
\usepackage[english]{babel}		% Set the languages; the last one is the main language
\usepackage{geometry}			% Allow to change the margin
\usepackage{setspace}			% Allow to change the interline with \begin{onehalfspaceng} ecc.
\usepackage{fancyhdr}			% Fancy style for page layout
\usepackage{afterpage} 			% Allow to load blank pages with the command '\afterpage{\null\thispagestyle{empty}\clearpage}' 
\usepackage{hyperref}			% Crea collegamenti ipertestuali rendendo cliccabili i riferimenti
\usepackage{url}
\usepackage{color}			% Allow to use colors
\usepackage{xcolor}			% manage colors
\usepackage{enumerate}			% Numbered lists
\usepackage{enumitem}			% Allow to manage the style of the list
\usepackage{graphicx}			% Images
\usepackage{epstopdf}			% allows to convert eps images to pdf for use with pdflatex
\usepackage{pstool}			% Allows to use psfrag with pdflatex (no latex compiler needed)
\usepackage{psfrag}
%\usepackage[normal]{subfigure}		% manage subfigures
\usepackage{subfig}		% manage subfigures
\usepackage{array}			% Tables
\usepackage{tabularx}			% allow to set the width of the whole table
\usepackage[bottom]{footmisc}		% To attach footnote at the end of the page
\usepackage{booktabs}			% is a must for professional-looking layout
\usepackage{longtable}			% is very popular for multi-page tables
\usepackage{caption}			% Captions
\usepackage{float}			% Manage float objects (image, table, ecc.)
\usepackage{subfloat}			% permette la sub numerazione degli oggetti mobili (immagini, tabelle, ecc.)
\usepackage{rotating}			% permette di ruotare immagini, tabelle, ecc. di 90° o di 270°
\usepackage{rotfloat}			% costruisce un ponte tra i pacchetti float e rotating
\usepackage{amsmath} 			% Equations
\usepackage{amssymb}			% Mathematical Symbols
\usepackage{cancel}                     % Simplification Cancellation
\usepackage{mleftright}
\usepackage{listings}                   % Codes
%\usepackage{vector}  			% Allows "\bvec{}" and "\buvec{}" for "blackboard" style bold vectors in maths
\usepackage[swapnames,norules,nouppercase]{frontespizio}
\usepackage[intoc]{nomencl}
\usepackage[acronym,toc,nomain,nopostdot,nonumberlist]{glossaries}
\usepackage[square, numbers, comma, sort&compress]{natbib}
\usepackage{verbatim}                   % Needed for the "comment" environment to make LaTeX comments
\usepackage{wrapfig}                 % Per inserire le figure contornate da testo
\usepackage{blindtext}
\geometry{a4paper,top=2.5cm,bottom=2.5cm,left=2cm,right=2cm,heightrounded,bindingoffset=5mm}
% -------------------------------------- Line-spacing --------------------------------------- %
\linespread{1.2} 
%\linespread{1.5} 
%\linespread{2} 
%\raggedbottom % disattiva lo 'stiracchiamento' del testo; va a favore di spazio bianco a fondo pagina
% ---------------------------------------- Numeration --------------------------------------- %
%\setcounter{secnumdepth}{3} % set the enumeration of the sections
%\setcounter{tocdepth}{3}    % set the enumeration of the sections in the table of contents
%--------------------------------------- Page layout -----------------------------------------%
%\pagestyle{fancy}                       % definisce lo stile di pagina aprendo la strada al pacchetto fancyhdr
%\renewcommand{\chaptermark}[1]{\markright{\chaptername\ \thechapter.\ #1}{}}    % ridefinisce la macro \rightmark per i capitoli
%\renewcommand{\sectionmark}[1]{\markright{\sectionname\ \thesection.\ #1}}      % ridefinisce la macro \rightmark per le sezioni
%\renewcommand{\sectionmark}[1]{\markright{\thesection.\ #1}}
%\lhead{Giacomo Della Posta}             % in alto a sinistra
%\chead{}                                % in alto al centro
%\rhead{\slshape \rightmark}             % in alto a destra (\rightmark contiene ...)
%\lfoot{Giacomo Della Posta}             % in basso a sinistra
%\cfoot{\thepage}                        % in basso al centro (\thepage per mettere il numero di pagina)
%\rfoot{}                                % in basso a destra
%\renewcommand{\headrulewidth}{0.4pt}    % spessore della linea di separazione in alto (0 per eliminare la linea)
%\renewcommand{\footrulewidth}{0.4pt}    % spessore della linea di separazione in basso (0 per eliminare la linea)
\fancypagestyle{plain}{%                                        % modifica dello stile predefinito plain
                        \fancyhf{}                              % cancella tutti i campi di  intestazione e pie di pagina
                        \fancyfoot[R]{\thepage}                 % mette al centro il numero di pagina
                        \renewcommand{\headrulewidth}{0pt}      % spessore della linea di separazione in alto (0 per eliminare la linea)
                        \renewcommand{\footrulewidth}{0.4pt}    % spessore della linea di separazione in basso (0 per eliminare la linea)
                        }                                       % modifica dello stile predefinito plain
\captionsetup{font=small,labelfont=bf,textfont=normalfont,tableposition=bottom,figureposition=bottom}
%\captionsetup{font=small,labelfont=bf,textfont=bf,tableposition=bottom,figureposition=bottom}
% ---------------------------------------- Itemize ------------------------------------------ %
%\setlist[itemize]{noitemsep, topsep=0pt} %Se non vuoi i pallini degli elenchi vuoti, allora commenta queste linee                                     
%\renewcommand{\labelitemi}{$\circ$}	% items i with empty bullets
%\renewcommand{\labelitemii}{$\bullet$}	% items ii with black bullets
\definecolor{sapienza}{RGB}{130,36,51} % example \definecolor{name}{model}{color-spec}
\definecolor{cust1}{RGB}{85,85,85}
\definecolor{cust2}{RGB}{212,212,212}
\makenomenclature % Generate the nomenclature
\makeglossaries % Generate the glossary
%\lstset{language=[90]Fortran,
%        inputpath=/home/luca/PhD/Thesis/Code,
%        basicstyle=\small\ttfamily, 
%        xleftmargin=0cm,
%        fontadjust=true,
%        keepspaces=true,
%        basewidth=0.5em,
%        breakatwhitespace=false,           % sets if automatic breaks should only happen at whitespace
%        breaklines=false,                  % sets automatic line breaking
%        captionpos=b,                     % b=bottom
%        backgroundcolor=\color{white},
%        identifierstyle=\color{black},
\bibliographystyle{abbrv} % abbrv --> [1], [2], ecc.
%\bibliographystyle{unsrtnat} 

% DEDICATION
%
% The dedication environment makes sure the dedication gets its
% own page and is set out in verse format.

\newenvironment{dedication}
{
  \phantom{.}
  \vspace{13cm}
  \begin{quote} \begin{flushright}}
{\end{flushright} \end{quote}}
\usepackage{calligra}