\thispagestyle{plain}			% Supress header 
\setlength{\parskip}{0pt plus 1.0pt}
\status{done}
\section*{Abstract}
In this thesis, we explore various aspects of particle physics research at the Paul Scherrer Institute (PSI) in Switzerland.
The introductory chapter provides an overview of the Standard Model at low energies, delving into some aspects of symmetry violation, and discussing phenomena like muon decay and the Electric Dipole Moment (EDM).
We then delve into the experimental landscape, outlying the current status of charged lepton flavor violation (cLFV) and EDM experiments.
This is necessary to give context to this thesis work, which focuses on the muon Electric Dipole Moment (muEDM) and the Muon to E Gamma conversion (MEG II).
These two projects are in different parts of their life-cycle: muEDM is still in fast development, typical of a young experiment, while MEG II is a running and well-known experiment in the last years of data-taking.\\

\noindent
In Ch.~2, we will explore the muEDM experiment, detailing the concept of EDM, current limits, and the frozen spin technique. 
The chapter provides a description of different subsystems and their current status, emphasizing the progress made during the last few years. 
In Ch.~3, we will discuss the concept of scintillation and its implementation in \textsc{Geant4}, then move to the Entrance and Telescope detectors developed for this experiment. 
Results from various beam times are presented, demonstrating satisfactory outcomes in line with muEDM requirements, particularly concerning the Entrance detector with multiple readouts and the TOF measurements.
The last of these chapters (Ch.~4) focuses on the design of the scintillating fiber detector for positron tracking in the muEDM experiment. 
This step is cardinal in demonstrating the frozen spin technique and measuring the EDM. 
The chapter outlines a desirable detector choice and explores potential designs and improvements: from the design included in the proposal of 2022 to the Cylindrical Helical Tracker (CHeT) and radial geometries under study.\\

\noindent
The second triplet of chapters shifts focus to the MEG II experiment.
In Ch.~5, we will provide insights into the MEG II apparatus and the different sub-systems, with particular emphasis on the Cockcroft–Walton accelerator, detailing the work with the machine and the hands-on experiences during the maintenance in 2022.
Ch.~6 is dedicated to the Charge EXchange calibration of the Liquid Xenon Calorimeter. In particular the functioning principle and the design history of the Liquid Hydrogen target.
In the last (Ch.~7) we will discuss the X17 search within the MEG II apparatus, detailing experiments conducted at ATOMKI, presenting ongoing analyses, data-taking campaigns, recent tests, and plans to enhance sensitivity to this anomaly.\\

\noindent
Overall, this thesis presents a comprehensive exploration of particle physics research at PSI, covering experimental developments in both the muEDM and MEG II projects, from early-stage design to advanced data analysis.
On a personal level, this work exposed me to many different topics: from the \textsc{Geant4} simulations for muEDM detectors and tracker to the cryogenics of the LH2 target, from the usage and maintenance of a CWz to the calibrations of a BGO calorimeter.

% KEYWORDS (MAXIMUM 10 WORDS)
\vfill
\begin{center}
\begin{minipage}{0.6\textwidth}
\underline{Keywords}: muon, EDM, muEDM, scintillators, \gf, cLFV, MEG II, Cockcroft-Walton, Liquid Hydrogen, X17.
\end{minipage}
\end{center}

\thispagestyle{empty}
\mbox{}